\section{Ausgangssituation}
Die Grundlage für dieses Projekt bildet der sogenannte Vierfachvordruck, ein 
internes Kommunikationsdokument des technischen Hilfswerks (THW). Es wird in 
der mobilen Einsatzzentrale des THW, der Fachgruppe Führung und Kommunikation 
(FGr FK) eingesetzt. Die FGr FK nutzt das Dokument, um ein- und ausgehende 
Nachrichten abzufassen und so z.B. Einsatzaufträge für Einheiten, eingehende 
Lagemeldungen oder Materialanforderungen abzuarbeiten. Der Vierfachvordruck 
ist eine Papier-Vorlage mit dreifachem Durchschlag. Zwei der Durchschläge 
werden an die zuständigen Personen innerhalb der FGr FK verteilt. Der dritte 
Durchschlag dient der Protokollierung. Im Angesicht heutiger Technologien 
erscheint dieses Verfahren nicht mehr zeitgemäß. Mithilfe einer Software wäre 
es möglich, den Prozess digital durchzuführen. Dadurch müssten keine 
handschriftlichen Dokumente verfasst und verteilt werden und das Verfahren 
könnte beschleunigt werden. Alle Nachrichten und Nachrichtenverläufe ließen 
sich leicht in einer Datenbank archivieren. Ein weiterer Vorteil wäre der 
Entfall von Mehraufwand durch schlecht lesbare Handschrift. Außerdem ließe 
sich die Verwaltung verschiedener Dokumente am Arbeitsplatz übersichtlicher 
gestalten, etwa durch ein digitales Postfach.