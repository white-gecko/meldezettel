\section{Ausgangssituation}
Die Grundlage für die Projekt bildet der sogenannte 4-Fach-Vordruck, ein internes Kommunikationsdokument des technischen Hilfswerks (THW). Es wird in der mobilen Einsatzzentrale des THW, der Fachgruppe Führung und Kommunikation (FGr FK) eingesetzt. Die FGr FK nutzt das Dokument, um ein- und ausgehende Nachrichten abzufassen und so z.B. Einsatzaufträge für Einheiten, eingehende Lagemeldungen oder Materialanforderungen abzuarbeiten. Der 4-Fach-Vordruck ist eine Papier-Vorlage mit dreifachem Durchschlag. Zwei der Durchschläge werden an die zuständigen Personen innerhalb der FGr FK verteilt; der dritte dient der Protokollierung. Im Angesicht heutiger Technologien ist dieses Verfahren nicht mehr zeitgemäß. Mithilfe einer Software wäre es möglich, diesen Prozess digital durchzuführen. Dadurch müssten keine handschriftlichen Dokumente verfasst und verteilt werden. Ein Vorteil wäre der Entfall von Mehraufwand durch schlecht lesbare Handschrift. Außerdem müssten die Zettel nicht mehr durch die Bearbeiter innerhalb der Zentrale verteilt werden, diese müssten somit nicht ständig ihre Posten verlassen. Weiterhin ließe sich die Verwaltung verschiedener Dokumente am jeweiligen Arbeitsplatz übersichtlicher gestalten, etwa durch ein digitales Postfach.