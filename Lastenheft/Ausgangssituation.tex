\subsection{Ausgangssituation}
Die Grundlage für die Projekt bildet ein internes Kommunikationsdokument des technischen Hilfswerks (THW). Es wird in der mobilen Einsatzzentrale des THW, der Fachgruppe Führung und Kommunikation (FGr FK) eingesetzt. Die FGr FK nutzt das Dokument, um ein- und ausgehende Nachrichten abzufassen und so z.B. Einsatzaufträge für Einheiten, eingehende Lagemeldungen oder Materialanforderungen abzuarbeiten. Das Dokument wird als 4-Fach-Vordruck bezeichnet und ist eine Papier-Vorlage mit dreifachem Durchschlag. Die Durchschläge werden an die zuständigen Personen innerhalb der FGr FK verteilt; einer dient der Protokollierung. Im Angesicht heutiger Technologien ist dieses Verfahren nicht mehr zeitgemäß. Mithilfe einer Software wäre es möglich, diesen Prozess digital durchzuführen. Dadurch müssten keine handschriftlichen Dokumente verfasst und verteilt werden. Ein Vorteil wäre der Entfall von Mehraufwand durch schlechte Handschrift. Außerdem müssten die Zettel nicht mehr durch die Bearbeiter innerhalb der Zentrale verteilt werden, diese müssten nicht ständig ihre Posten verlassen. Weiterhin ließe sich die Verwaltung verschiedener Dokumente am jeweiligen Arbeitsplatz durch eine digitale Eingangsliste übersichtlicher gestalten.