\subsection{/LF0500/ Speicherung im RDF-Format}
Die vom Verfasser im Vierfachvordruck eingegebenen Informationen, sollen im RDF Datenmodell mithilfe des im QuitStore verfügbaren SPARQL-Endpoints auf einem lokalen Server abgespeichert werden. Demnach werden die Eingaben, z.B. 'Dokument hat Absender: Nathanael Arndt' in folgender Form dargestellt:
\begin{flushleft}
\quad \quad \quad Subjekt Vierfachvordruck (repräsentiert durch URI) \\
\quad \quad \quad Prädikat `hatAbsender' \\
\quad \quad \quad Objekt `Natanael Arndt' (repräsentiert durch URI oder Name)
\end{flushleft}
Die anderen Stationen, die bisher Kopien des Vierfachvordrucks erhalten haben, müssen dann auf die Eintragungen des Verfassers zugreifen können. Dies wird direkt durch den SPARQL-Endpoint ermöglicht, welcher eine Oberfläche für gezielte SPARQL-Anfragen an das Datenmodell bietet.