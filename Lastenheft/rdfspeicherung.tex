\subsection{Speicherung im RDF-Format}
Die vom Verfasser im Vierfachvordruck eingegebenen Informationen, sollen im RDF Datenmodell auf einem lokalen Server abgespeichert werden. Demnach werden die Eingaben, z.B. 'Absender: Nathanael Arndt' wie in `example-1.ttl' vorgegeben in der Form 
\begin{center}<VierfachvordruckID> thw:absender `Nathanael Arndt'. \end{center}
in einer .ttl oder .owl Textdatei abgespeichert werden. Auf diese Art werden alle Daten die gesamelt werden innerhalb einer Datei gespeichert. \\
Die anderen Stationen, die bisher Kopien des Vierfachvordrucks erhalten haben, müssen nun digital auf die Eintragungen des Verfassers zugreifen können. Also müssen gezielte SPARQL-Anfragen implementiert werden, die die Informationen des jeweiligen Viefachvordrucks aus der Datei auslesen und an die Ausgabe übergeben.