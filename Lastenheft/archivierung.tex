\subsection{/LF0400/ Archivierung}
Es ist von immenser Bedeutung für das Technische Hilfswerk, nach einiger Zeit wieder auf die Angaben eines Vierfachvordrucks
zugreifen zu können. Momentan wird dies ermöglicht, indem eine Kopie des Zettels im Archiv geordnet abgelegt wird. Da wir das manuelle Ausfüllen eines Vierfachvordrucks mit unserer Applikation abschaffen wollen, müssen wir dafür sorgen, dass die Archivierung trotzdem noch aktiv gehalten werden kann. Das heißt, es ist unbedingt erforderlich, dass die gespeicherten Dokumente ausdruckbar sind. Dabei sollte die Darstellungsform der ursprünglichen Zettel möglichst beibehalten werden, um die Archivierung einheitlich zu halten. Außerdem sollen die Eingaben auch noch längerfristig auf einem Server gespeichert werden, um u.a. einen noch schnelleren Zugriff zu ermöglichen. Dieser Server ist derselbe wie der, über den unser Programm später laufen soll, ist somit also nur lokal zugreifbar.