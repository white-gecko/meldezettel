\newglossaryentry{rdf} {
    name = RDF,
    description = {
        Kurz für Resource Description Framework. RDF ist ein Datenmodell, 
        welches alle Daten in Triples der Form Subjekt,
        Prädikat, Objekt dargestellt. Ein Subjekt Leipzig wäre beispielsweise 
        durch das Prädikat hatHochschule mit dem Objekt Universität Leipzig 
        verbunden. Durch diese Schreibweise bildet sich eine Baumstruktur, in 
        der Entitäten durch Relationen miteinander verknüpft sind
    }
}

\newglossaryentry{sparql} {
    name = SPARQL,
    description = {
        Kurz für Semantic Protocol and RDF Query Language. SPARQL ist eine 
        Anfragesprache, mit der man Informationen aus einem RDF 
        Datenmodell extrahieren und modifizieren kann
    }
}

\newglossaryentry{Framework} {
    name = Framework,
    description = {
        Ein Framework stellt dem Softwareentwickler einen Rahmen mit festen 
        Regeln und wiederverwendbare Strukturen zur Verfügung. Insbesondere 
        definiert es den Kontrollfluss der Anwendung und die Schnittstellen der 
        Klassen. Im Gegensatz zur Arbeit mit Klassenbibliotheken werden nicht 
        bereits vorhandene Klassen und Funktionen verknüpft und verwendet um den 
        Ansprüchen des Projekts zu genügen, stattdessen werden selbständig 
        Funktionen und Klassen in einem vorgegebenen Rahmen implementiert, 
        welche dann vom Framework genutzt werden um vordefinierte Aufgaben zu 
        erledigen. Woraus sich auch ergibt das Frameworks stark 
        Anwendungsspezifisch sind
    }
}

\newglossaryentry{Unit Test} {
    name = Unit Test,
    description = {
        Unit-Tests dienen dazu, diskrete Einheiten des Quellcodes auf ihre 
        Funktionalität hin zu überprüfen. Hierzu werden im Vorfeld Testfälle mit 
        möglichst umfassenden Eingabeparametern und den zu erwartenden Output 
        definiert. Der Test besteht dann daraus, dass der der Output, den 
        die Einheit aus dem definierten Input produziert, mit dem erwarteten 
        Output verglichen wird. Testing-Frameworks helfen beim erstellen und 
        automatisierten Durchführen dieser Tests
    }
}

\newglossaryentry{Deployment} {
    name = Deployment,
    description = {
        Bezeichnet den Prozess der Installation und Konfiguration von Software 
        auf Computern. Dieselbe Software kann mehrmals `deployed' werden, z.B. 
        wenn eine neue Version der Software herausgegeben wurde
    }
}

\newglossaryentry{Model-View-Controller} {
    name = Model-View-Controller,
    description = {
        MVC ist ein Muster zur Implementierung von User-Interfaces, welches 
        Quellcode in drei unabhängigen Komponenten organisiert. Dies soll die 
        interne Repräsentation von Informationen und deren Verarbeitung von der 
        Art wie diesem dem Nutzer präsentiert werden trennen. Dies erhöht die 
        Wiederverwendbarkeit von Code und erlaubt die parallele Entwicklung der 
        verschiedenen Teile. Das Model ist hierbei die zentrale Komponente, 
        welche das Problem beschreibt. Es managt direkt die Daten und Logik des 
        Programms. Die View-Komponente besteht aus der Darstellung der Daten 
        z.B. durch Tabellen oder Diagramme und ermöglicht es verschiedene Views 
        für verschiedene Nutzer zu definieren.
        Der Controller akzeptiert User-Input der dann in die entsprechenden 
        Befehle für Model und View umgewandelt wird
    }
}

\newglossaryentry{Markdown} {
    name = Markdown,
    description = {
        Markdown ist eine einfache Auszeichnungssprache, und lässt sich u.a. 
        direkt in HTML übersetzen. Die Ausdrücke sind einfach zu lesen, zu 
        schreiben und zu bearbeiten. 
        Verwendet wird Markdown z.B. auf GitHub oder auch für Jekyll
    }
}

\newglossaryentry{HTML} {
    name = HTML,
    description = {
        HTML steht für Hypertext Markup Language und ist eine Sprachkonvention, 
        mit der digitale Dokumente für die Interpretation durch Web-Browser 
        semantisch strukturiert werden. HTML wurde entwickelt, um Informationen 
        digital austauschen zu können, ohne diese mehrmals umformen und anpassen 
        zu müssen. Ein HTML-Dokument besteht aus einem Doctype-Tag 
        (Dokumenttypdeklaration), Head 
        (Meta Informationen) und Body (der Inhalt)
    }
}

\newglossaryentry{CSS} {
    name = CSS,
    description = {
        CSS steht für Cascading Style Sheets und ist eine Gestaltungssprache, 
        über die sich die Darstellung von Informationen steuern lässt. Im 
        Gegensatz zu HTML wird sich dabei nicht auf den Inhalt bezogen, sondern 
        allein auf das Design
    }
}

\newglossaryentry{Skriptsprache} {
    name = Skriptsprache,
    description = {
        Eine Programmiersprache, die sich vor allem durch Interpreterausführung 
        und oftmals entfallenden Deklarationszwang auszeichnet
    }
}

\newglossaryentry{JavaScript} {
    name = JavaScript,
    description = {
        JavaScript ist eine Skriptsprache, die ursprünglich dazu diente 
        Benutzerinteraktionen mit Webseiten auszuwerten und daraufhin Inhalte 
        anzupassen, nachzuladen oder zu erzeugen. Dieser Anwendungsbereich wurde 
        inzwischen durch unzählige Bibliotheken stark erweitert. JavaScript ist 
        eine dynamisch typisierte, objektorientierte, klassenlose Skriptsprache
    }
}

\longnewglossaryentry{Continuous-Integration}
{
    name = Continuous Integration
}
{
    Der Begriff Continuous Integration (CI) beschreibt eine 
    Softwareentwicklungspraxis, in der alle beteiligten Entwickler eines 
    Projektes häufig (ein- bis mehrmals am Tag) ihren aktuellen
    Arbeitsstand (meistens verwirklicht in einem dezidierten Branch) in den 
    Hauptstand (Masterbranch) integrieren.

    Um dieses Ziel zu erreichen werden meist folgende Tools verwendet:

    \begin{itemize}
        \item Version Control System (z.B. Git)
        \item automatisierte Tests (z.B. Unit- oder Integrationtests)
        \item CI-Server (z.B. Github CI)
    \end{itemize}

    Die direkten Vorteile der CI sind zum einen der minimale Integrationsaufwand 
    sowie die fähigkeit, jederzeit eine stabile version veröffentlichen zu 
    können.
    Der minimale Integrationsaufwand ergibt sich durch die Häufigkeit,
    mit der die Entwickler ihren Arbeitsstand integrieren. Durch die
    Bedingung, dass vor jeder Integration alle automatisierten Testfälle ohne 
    Fehlermeldungen durchlaufen müssen ergibt sich ein relativ stabiler 
    Hauptstand, der jederzeit als Grundlage für neue Features dienen kann.
    Bugs die nicht durch Tests erkannt wurden und erst im release auffallen, 
    können dank der kompakten Natur kurzer Integrationszyklen schnell erkannt 
    und behoben werden.
    Kommunikation ist insofern ein wichtiger Aspekt der CI, als dass jederzeit 
    alle Entwickler den Zustand der Software kennen können sollten. Hierbei 
    hilft der Einsatz eines CI-Servers, der beispielsweise jederzeit wenn Code 
    in den Hauptstand integriert werden soll, den kompletten Build Prozess 
    nachvollzieht, alle Tests durchlaufen lässt und entsprechend alle
    auftauchenden Fehlermeldungen an das Team weitergibt
}

\newglossaryentry{Git} {
    name = Git,
    description = {
        Git ist eine freie Software zur verteilten Versionsverwaltung
    }
}

\newglossaryentry{QuitStore} {
    name = QuitStore, 
    description = {
        Software die Git-Versionierung für sogenannte `Named Graphs' (ein 
        Schlüsselkonzept des Semantic Web das bewirkt, dass eine Menge an RDF 
        Ausdrücken über eine URI identifiziert werden) ermöglicht
    }
}