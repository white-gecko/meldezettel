    \newglossaryentry{rdf} {
        name = RDF,
        description = {
            Kurz für Resource Description Framework, ist ein Datenmodell, 
            das stark von der üblichen Vorstellung von Daten in Tabellen 
            abweicht. Alle Daten werden immer in Triples der Form Subjekt,
            Prädikat, Objekt dargestellt, z.B. ein Subjekt Leipzig ist durch 
            das Prädikat hatHochschule mit
            dem Objekt Universität Leipzig verbunden. Durch diese 
            Schreibweise bildet sich bei Verwendung einer großen Menge von 
            Triples eine riesige Baumstruktur auf, in der viele Entitäten 
            durch Relationen miteinander verknüpft sind.
        }
    }

    \newglossaryentry{sparql} {
        name = SPARQL,
        description = {
            Kurz für Semantic Protocol and RDF Query Language, ist wie SQL für 
            tabellenbasierte Datenbanken, eine Anfragesprache, mit der man gezielt 
            Informationen aus einem RDF Datenmodell extrahieren kann.
        }
    }


    \newglossaryentry{Framework} {
        name = Framework,
        description = {
            Ein Framework stellt dem Softwareentwickler einen Rahmen mit festen Regeln und 
            wiederverwendbare Strukturen zur Verfügung. Insbesondere definiert es den 
            Kontrollfluss der Anwendung und die Schnittstellen der Klassen. Im Gegensatz 
            zur Arbeit mit Klassenbibliotheken werden nicht bereits vorhandene Klassen und 
            Funktionen verknüpft und verwendet um den Ansprüchen des Projekts zu genügen, 
            stattdessen werden selbständig Funktionen und Klassen in einem vorgegebenen 
            Rahmen implementiert, welche dann vom Framework genutzt werden um 
            vordefinierte Aufgaben zu erledigen. Woraus sich auch ergibt das Frameworks 
            stark Anwendungsspezifisch sind.
        }
    }


    \newglossaryentry{Modultest} {
        name = Modultest,
        description = {
            Modultests sind ein Bestandteil des Softwaretests und beschreiben das 
            automatische Testen von Softwaremodulen auf deren korrekte Funktionalität. 
            Hierfür müssen im Vorfeld Testfälle mit möglichst umfassenden 
            Eingabeparametern und den zu erwartenden Ausgabeparametern definiert werden, 
            mit welchen das Modul - bspw. unter Verwendung eines Testing Frameworks -
            auf Funktionalität geprüft wird.    
        }
    }


    \newglossaryentry{Deployment} {
        name = Deployment,
        description = {
            Bezeichnet die Visualisierung und Definition von Schritten und der 
            Rollenverteilung in einem Prozess. Die aus Deployment resultierenden 
            Flowcharts stellen eine Reihenfolge von Aktionen/Schritten, sowie die 
            Interaktion zwischen verschiedenen Personen und Gruppen dar. Der große Vorteil 
            dieses Modells liegt bei Hervorhebung von Ineffizienz, Dopplung und unnötigen
            Abläufen.
        }
    }


    \newglossaryentry{Model-View-Controller} {
        name = Model-View-Controller,
        description = {
            MVC ist ein Muster zur Implementierung von User-Interfaces, welches ein 
            Programm in 3 Teile zerlegt. Dies soll die interne Repräsentation von 
            Informationen und deren Verarbeitung von der Art wie diesem dem Nutzer 
            präsentiert werden trennen. Dies erhöht die Wiederverwendbarkeit
            von Code und erlaubt die parallele Entwicklung der verschiedenen Teile. Das 
            Model ist hierbei die zentrale Komponente, welche das Problem beschreibt. Es 
            managt direkt die Daten und Logik des Programms. Die View-Komponente besteht 
            aus der Darstellung der Daten z.B. durch Tabellen oder Diagramme und 
            ermöglicht es verschiedene Views für verschiedene Nutzer zu definieren.
            Der Controller akzeptiert User-Input der dann in die entsprechenden Befehle 
            für Model und View umgewandelt wird. Ursprünglich wurde MVC vor allem für die 
            Entwicklung von Desktop GUIs benutzt, allerdings hat sich das Muster auch 
            mitlerweile für die Entwicklung von Web- und Mobile- Applikationen bewährt.
        }
    }


    \newglossaryentry{Java} {
        name = Java,
        description = {
            Java ist eine allzweck Programmiersprache, welche Klassen- und 
            Objektorientiert ist und sich vor allem durch ihre Plattformunabhängigkeit 
            auszeichnet. Der Quellcode wird hierfür in bytecode kompiliert, der dann auf 
            jeder Java virtual machine unabhängig von System oder Architektur zum laufen 
            gebracht werden kann.
        }
    }


    \newglossaryentry{Markdown} {
        name = Markdown,
        description = {
            Markdown ist eine einfache Auszeichnungssprache, und lässt sich unter anderem 
            direkt in HTML übersetzen. Die Ausdrücke sind einfach zu lesen, zu schreiben 
            und zu bearbeiten. 
            Verwendet wird Markdown z.B. auf GitHub oder auch für Jekyll
        }
    }


    \newglossaryentry{HTML} {
        name = HTML,
        description = {
            HTML steht für Hypertext Markup Language und ist eine Sprachkonvention, mit 
            der digitale Dokumente für die Interpretation durch Web-Browser semantisch 
            strukturiert werden. HTML wurde von Forschern entwickelt, um Informationen 
            digital austauschen zu können, ohne diese mehrmals umformen und anpassen zu 
            müssen. Ein HTML-Dokument besteht aus einem Doctype-Tag 
            (Dokumenttypdeklaration), Head 
            (Meta Informationen) und Body (der Inhalt).
        }
    }


    \newglossaryentry{CSS} {
        name = CSS,
        description = {
            CSS steht für Cascading Style Sheets ist eine Gestaltungssprache, über die 
            sich die Darstellung von Informationen steuern lässt. Im Gegensatz zu HTML 
            wird sich dabei nicht auf den Inhalt bezogen, sondern allein auf das Aussehen.
        }
    }


    \newglossaryentry{Skriptsprache} {
        name = Skriptsprache,
        description = {
            Eine Programmiersprache, die vor allem durch Interpreterausführung und oftmals 
            entfallenden Deklarationszwang auszeichnen.
        }
    }



    \newglossaryentry{JavaScript} {
        name = JavaScript,
        description = {
            JavaScript ist eine Skriptsprache, die dazu dient, Benutzerinteraktionen mit 
            Webseiten auszuwerten und daraufhin Inhalte anzupassen, nachzuladen oder zu 
            erzeugen. Dadurch sollen die Funktionalitäten von HTML und CSS erweitert 
            werden. JavaScript ist eine dynamisch typisierte, objektorientierte, 
            klassenlose Skriptsprache.
        }
    }


   \newglossaryentry{Continuous-Integration} {
       name = Continuous Integration,
       description = {
           Der Begriff Continuous Integration (CI) beschreibt eine Softwareentwicklungspraxis,
           in der alle beteiligten Entwickler eines Projektes häufig (ein- bis mehrmals am
           Tag) ihren aktuellen Arbeitsstand (meistens verwirklicht in einem dezidierten
           Branch) in den Hauptstand (Masterbranch)
           integrieren. 
       }
   }

    \newglossaryentry{Git} {
        name = git,
        description = {
            Git ist eine freie Software zur verteilten Versionsverwaltung.
        }
    }
    
    \newglossaryentry{QuitStore} {
        name = QuitStore, 
        description = {
            Software die Git-Versionierung für sogenannte "Named Graphs" (ein Schlüsselkonzept 
            des Semantic Web das bewirkt, dass eine Menge an RDF Ausdrücken über eine URI 
            identifiziert werden) ermöglicht
        }
    }

