\section{Glossar}

    \subsection{RDF}
    Kurz für Resource Description Framework, ist ein Datenmodell, 
    das stark von der üblichen Vorstellung von Daten in Tabellen 
    abweicht. Alle Daten werden immer in Triples der Form Subjekt,
    Prädikat, Objekt dargestellt, z.B. ein Subjekt Leipzig ist durch 
    das Prädikat hatHochschule mit
    dem Objekt Universität Leipzig verbunden. Durch diese 
    Schreibweise bildet sich bei Verwendung einer großen Menge von 
    Triples eine riesige Baumstruktur auf, in der viele Entitäten 
    durch Relationen miteinander verknüpft sind.

    \subsection{SPARQL}
    Kurz für Semantic Protocol and RDF Query Language, ist wie SQL für 
    tabellenbasierte Datenbanken, eine Anfragesprache, mit der man gezielt 
    Informationen aus einem RDF Datenmodell extrahieren kann.

    \subsection{Framework}
    Ein Framework stellt dem Softwareentwickler einen Rahmen mit festen Regeln und 
    wiederverwendbare Strukturen zur Verfügung. Insbesondere definiert es den 
    Kontrollfluss der Anwendung und die Schnittstellen der Klassen. Im Gegensatz 
    zur Arbeit mit Klassenbibliotheken werden nicht bereits vorhandene Klassen und 
    Funktionen verknüpft und verwendet um den Ansprüchen des Projekts zu genügen, 
    stattdessen werden selbständig Funktionen und Klassen in einem vorgegebenen 
    Rahmen implementiert, welche dann vom Framework genutzt werden um 
    vordefinierte Aufgaben zu erledigen. Woraus sich auch ergibt das Frameworks 
    stark Anwendungsspezifisch sind.

    \subsection{Modultest}
    Modultests sind ein Bestandteil des Softwaretests und beschreiben das 
    automatische Testen von Softwaremodulen auf deren korrekte Funktionalität. 
    Hierfür müssen im Vorfeld Testfälle mit möglichst umfassenden 
    Eingabeparametern und den zu erwartenden Ausgabeparametern definiert werden, 
    mit welchen das Modul - bspw. unter Verwendung eines Testing Frameworks -
    auf Funktionalität geprüft wird.    

    \subsection{Deployment}
    Bezeichnet die Visualisierung und Definition von Schritten und der 
    Rollenverteilung in einem Prozess. Die aus Deployment resultierenden 
    Flowcharts stellen eine Reihenfolge von Aktionen/Schritten, sowie die 
    Interaktion zwischen verschiedenen Personen und Gruppen dar. Der große Vorteil 
    dieses Modells liegt bei Hervorhebung von Ineffizienz, Dopplung und unnötigen
    Abläufen.

    \subsection{Model-View-Controller}
    MVC ist ein Muster zur Implementierung von User-Interfaces, welches ein 
    Programm in 3 Teile zerlegt. Dies soll die interne Repräsentation von 
    Informationen und deren Verarbeitung von der Art wie diesem dem Nutzer 
    präsentiert werden trennen. Dies erhöht die Wiederverwendbarkeit
    von Code und erlaubt die parallele Entwicklung der verschiedenen Teile. Das 
    Model ist hierbei die zentrale Komponente, welche das Problem beschreibt. Es 
    managt direkt die Daten und Logik des Programms. Die View-Komponente besteht 
    aus der Darstellung der Daten z.B. durch Tabellen oder Diagramme und 
    ermöglicht es verschiedene Views für verschiedene Nutzer zu definieren.
    Der Controller akzeptiert User-Input der dann in die entsprechenden Befehle 
    für Model und View umgewandelt wird. Ursprünglich wurde MVC vor allem für die 
    Entwicklung von Desktop GUIs benutzt, allerdings hat sich das Muster auch 
    mitlerweile für die Entwicklung von Web- und Mobile- Applikationen bewährt.

    \subsection{Java}
    Java ist eine allzweck Programmiersprache, welche Klassen- und 
    Objektorientiert ist und sich vor allem durch ihre Plattformunabhängigkeit 
    auszeichnet. Der Quellcode wird hierfür in bytecode kompiliert, der dann auf 
    jeder Java virtual machine unabhängig von System oder Architektur zum laufen 
    gebracht werden kann.

    \subsection{Markdown}
    Markdown ist eine einfache Auszeichnungssprache, und lässt sich unter anderem 
    direkt in HTML übersetzen. Die Ausdrücke sind einfach zu lesen, zu schreiben 
    und zu bearbeiten. 
    Verwendet wird Markdown z.B. auf GitHub oder auch für Jekyll

    \subsection{HTML}
    HTML steht für Hypertext Markup Language und ist eine Sprachkonvention, mit 
    der digitale Dokumente für die Interpretation durch Web-Browser semantisch 
    strukturiert werden. HTML wurde von Forschern entwickelt, um Informationen 
    digital austauschen zu können, ohne diese mehrmals umformen und anpassen zu 
    müssen. Ein HTML-Dokument besteht aus einem Doctype-Tag 
    (Dokumenttypdeklaration), Head 
    (Meta Informationen) und Body (der Inhalt).

    \subsection{CSS}
    CSS steht für Cascading Style Sheets ist eine Gestaltungssprache, über die 
    sich die Darstellung von Informationen steuern lässt. Im Gegensatz zu HTML 
    wird sich dabei nicht auf den Inhalt bezogen, sondern allein auf das Aussehen.

    \subsection{Skriptsprache}
    Eine Programmiersprache, die vor allem durch Interpreterausführung und oftmals 
    entfallenden Deklarationszwang auszeichnen.


    \subsection{JavaScript}
    JavaScript ist eine Skriptsprache, die dazu dient, Benutzerinteraktionen mit 
    Webseiten auszuwerten und daraufhin Inhalte anzupassen, nachzuladen oder zu 
    erzeugen. Dadurch sollen die Funktionalitäten von HTML und CSS erweitert 
    werden. JavaScript ist eine dynamisch typisierte, objektorientierte, 
    klassenlose Skriptsprache.

    \subsection{Continuous Integration}
    Der Begriff Continuous Integration (CI) beschreibt eine Softwareentwicklungspraxis, in der alle beteiligten Entwickler eines Projektes häufig (ein- bis mehrmals am Tag) ihren aktuellen Arbeitsstand (meistens verwirklicht in einem dezidierten Branch) in den Hauptstand (Masterbranch)
    integrieren. Um dieses Ziel zu erreichen werden meist folgende Tools im Entwicklungsprozess verwendet:

        \begin{itemize}
            \item Version Control System (z.B. Git)
            \item automatisierte Unit-Tests
            \item CI-Server (z.B. Github CI)
        \end{itemize}

    Die direkten Vorteile der CI sind zum einen der minimale Integrationsaufwand 
    sowie die fähigkeit, jederzeit eine stabile version veröffentlichen zu können. 
    Der minimale Integrationsaufwand ergibt sich durch die Häufigkeit, mit der die 
    Entwickler ihren Arbeitsstand integrieren. Durch die Bedingung dass vor jeder 
    Integration alle automatisierten Testfälle ohne Fehlermeldungen
    durchlaufen müssen ergibt sich ein relativ stabiler Hauptstand, der jederzeit 
    als Grundlage für neue Features dienen kann. Bugs die nicht durch Unit-Tests 
    erkannt wurden und erst im release auffallen, können dank der kompakten Natur 
    kurzer Integrationszyklen schnell erkannt und behoben werden. Kommunikation 
    ist insofern ein wichtiger Aspekt der CI, als dass jederzeit
    alle Entwickler den Zustand der Software kennen können sollten. Hierbei hilft 
    der Einsatz eines CI-Servers, der beispielsweise jederzeit wenn Code in den 
    Hauptstand integriert werden soll, den kompletten Build Prozess nachvollzieht, 
    alle Tests durchlaufen lässt und entsprechend alle auftauchenden 
    Fehlermeldungen an das Team weitergibt

    \subsection{git}
    Git ist eine freie Software zur verteilten Versionsverwaltung. Die Entwicklung 
    wurde von Linus Torvalds angestoßen. Gegenüber anderen 
    Versionskontrollsystemen unterscheidet es sich allerdings in einigen Aspekten:

        \begin{itemize}
            \item Nicht lineare Entwicklung:\\
            D.h. jeder Entwickler hat sein eigenes Repository und jedes Feature
            wird in seinem eigenen Branch entwickelt.
            \item "Branching"(Erstellen neuer Entwicklungszweige) und
            \item "Merging"(Verschmelzen mind. zweier Entwicklungszweige) sind integrale Bestandteile der Arbeit mit Git. Des Weiteren werden nativ Tools zur Visualisierung der nicht linearen Geschichte bereit gestellt.
            \item Kein zentraler Server:\\
            Jeder Benutzer hat eine lokale Kopie des gesamten Repository und der Versionsgeschichte. Das bietet zum einen Sicherheit (Datensicherung) und zum anderen ist für die meisten Tätigkeiten so kein Netzwerkzugriff erforderlich. Zwischen den lokalen und entfernten Repositories gibt es zwar keinen technischen Unterschied, aber meist gibt es ein offizielles Repository.
            \item Kryptographische Sicherheit der Projektgeschichte:\\
            Die Projektgeschichte wird so gespeichert, dass der Hash-Code einzelner Commits immer auf der ganzen Geschichte basiert. Somit ist
            nachträgliche Manipulation nicht möglich, ohne den Hash-Code automatisch zu ändern.
            \item Unterstützung vieler Übertragungsprotokolle:\\
            Die Unterstützung diverser Übertragungsprotokolle wie HTTP, HTTPS, FTP, SSH und rsync bietet dem Nutzer große Freiheit seine Daten zu
            senden wie er es für richtig hält.
            \item Säubern des Repositories:\\
            Auch wenn man Daten/Aktionen aus den Repositories löscht/zurück-
            nimmt, so bleiben sie doch erstmal erhalten und können zu jedem Zeitpunkt wiederhergestellt werden, was Fehler beim entfernen von Dateien/Aktionen sehr schwer macht. Erst wenn man sich nach diesem Schritt sicher ist kann man die Daten explizit und für immer löschen.
        \end{itemize}
    
    \subsection{QuitStore}
    Software die Git-Versionierung für sogenannte "Named Graphs" (ein Schlüsselkonzept des Semantic Web das bewirkt, dass eine Mengen an RDF Ausdrücken über eine URI identifiziert werden) ermöglicht

\end{document}