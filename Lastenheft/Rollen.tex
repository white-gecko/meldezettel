\subsection{/LF0000/ Verschiedene Rollen}
Zum Zwecke der besseren Übersicht und Nachbildung des Workflows ist die Aufteilung in verschiedene Rollen, welche mit den entsprechenden Rollen des aktuellen Workflows übereinstimmen, notwendig. Realisiert wird das, indem beim Start der Applikation die jeweilige, gerade besetzte Position angegeben wird und dann im weiteren Arbeitsverlauf auch nur die Arbeitsschritte, z.B. das Ausfüllen bestimmter Felder des Vierfachvordrucks, von dieser Person ausgeführt werden können, wohingegen die anderen gesperrt sind. Dadurch wird die Integrität der einzelnen Rollen gewährleistet. Dabei wird die gerade besetzte Rolle stetig durch die Farbgebung der Oberfläche deutlich gemacht, nah an der Farbgebung der Kopie des Vierfachvordrucks gehalten, die man an dieser Position erhalten würde. Die Farbgebung kann sich allerdings auch variabel gestalten, z.B. wenn der Vierfachvordruck zur Korrektur nochmal zurückgeschickt wurde, wird er mit der Farbe des Arbeitsschritts anstelle der Farbe der Rolle dargestellt. 
