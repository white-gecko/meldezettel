\subsection{Vorprojekt}

    Der Fokus im Vorprojekt liegt auf der Erstellung der Benutzeroberfläche zur Erstellung eines Vierfachvordrucks. 
    Die GUI wird über den Browser aufgerufen und ermöglicht in der Startseite die Auswahl der entsprechenden Rolle über 
    ein Dropdown-Menü. In derselben Übersichtsseite sind die für die ausgewählte Rolle relevanten Informationen ersichtlich. 
    Jede Rolle hat einen eigenen View. Einen neuen Vordruck zu erstellen ist für jeden Nutzer möglich, dabei soll das 
    Design des Formulars im Vorprojekt weitestgehend fertig umgesetzt sein.
    
    Zu Demonstrationszwecken sollen einige fest voreingestellte Vordrucke zur Verfügung stehen. Neu erstellte Vordrucke
    bleiben zunächst nur temporär bestehen und gehen beim Verlassen der Seite verloren. Dadurch sollen Testnutzer
    in der Lage sein, einen ersten Überblick über den Prozess der Verfassens eines Vierfachvordrucks zu erhalten,
    ohne sich Gedanken über die Integrität der Daten machen zu müssen.

    Der Workflow liegt hierbei zunächst nicht im Fokus; dieser wird erst im fertigen Projekt ersichtlich sein, da hierzu
    ein funktionsfähiges Backend benötigt wird.
