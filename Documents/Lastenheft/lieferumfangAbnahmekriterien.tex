\section{Lieferumfang und Abnahmekriterien}
Das fertige Produkt umfasst eine voll funktionale Anwendung, welche innerhalb einer mobilen Einsatzzentrale die Arbeit mit dem Vierfachvordruck ersetzen kann. Aufgrund der großen Verantwortung des THW ist die Zuverlässigkeit der Software besonders wichtig. Um diese zu gewährleisten muss die Software und deren Module vor ihrer Auslieferung und während der Entwicklung vollumfänglichen, automatisierten Tests unterzogen werden. Wofür ein Testingframework Anwendung findet, welches Modultests und Integrationstest durchführt. Eine Testcoverage von 100 \% muss gewährleistet sein. Eine weitere Komponente ist das zuverlässige Speichern, Abfragen und Darstellen der ausgefüllten Vordrucke. Hierzu wird ein Datenbanksystem aufgesetzt, welches das RDF-Modell verwendet, SPARQL als Querylanguage unterstützt und die Quit-Store Speicher-Architektur nutzt. Die Darstellung orientiert sich stark am Original des Vierfachvordrucks und erfolgt über einen Browser. Um die Verlässlichkeit weiter zu steigern wird darüber hinaus das Prinzip der kontinuierlichen Integration angewandt. Somit kann eine gemeinsame, einfach erreichbare und verlässliche Codebasis für das gesamte Team geschaffen werden. Ein weiteres wichtiges Abnahmekriterium ist die hohe Gebrauchstauglichkeit, wobei insbesondere die schnelle Zugänglichkeit, die klare Trennung der Rollen und die Möglichkeit des effizienten und schnellen Arbeitens zu gewährleisten ist. Aus der geforderten Usability folgt auch eine entsprechende Performanz, welche das unterbrechungsfreie, parallele Arbeiten von bis zu elf Nutzern (sechs Sachgebietsbearbeiter, zwei Funker, ein Sichter, ein Lagekartenführer und ein Stabsleiter) garantiert, wobei als Hardware ein Raspberry Pi vorgesehen ist. Um die Akzeptanz weiter zu steigern, soll die Software in der Lage sein den Nutzern durch ihre Funktionalität weitere Aufgaben abzunehmen oder zumindest zu erleichtern. Dazu seien bspw. das Drucken der rechtlich belastbaren Unterlagen und die verschiedenen Sichten mit ihren Zugriffsbeschränkungen genannt. Im Hinblick auf die mögliche Weiterentwicklung ist auch die Wartbarkeit durch eine ausführliche, wohl strukturierte Dokumentation, sowie einheitlich formatierten und gut leserlichen Code möglichst komplikationsfrei zu gestalten. Die Kompatibilität muss in sofern gegeben sein, dass die Webseite auf unterschiedlichen Geräten gut erkenn- und bedienbar sein soll. Durch die feste Hardware ist die Portabilität weniger wichtig, trotzdem sollte die Möglichkeit bestehen ohne großen Aufwand auf ein anderes System zu migrieren. 