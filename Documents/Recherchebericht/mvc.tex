\subsection{Model-View-Controller}
MVC ist ein Muster zur Implementierung von User-Interfaces, welches ein Programm in 3
Teile zerlegt. Dies soll die interne Repräsentation von Informationen und deren Verarbeitung
von der Art wie diesem dem Nutzer präsentiert werden trennen. Dies erhöht die Wiederverwendbarkeit
von Code und erlaubt die parallele Entwicklung der verschiedenen Teile. Das Model ist hierbei
die zentrale Komponente, welche das Problem beschreibt. Es managt direkt die Daten und Logik
des Programms. Die View-Komponente besteht aus der Darstellung der Daten z.B. durch Tabellen
oder Diagramme und ermöglicht es verschiedene Views für verschiedene Nutzer zu definieren.
Der Controller akzeptiert User-Input der dann in die entsprechenden Befehle für Model und 
View umgewandelt wird. Ursprünglich wurde MVC vor allem für die Entwicklung von Desktop 
GUIs benutzt, allerdings hat sich das Muster auch mitlerweile für die Entwicklung von 
Web- und Mobile- Applikationen bewährt. \cite{mvcwiki}

