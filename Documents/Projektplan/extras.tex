\section{Extras}
    \subsection{Benutzerhandbuch}

    Um Erstnutzern einen schnellen Überblick über die zweckmäßige Nutzung der
    Software zu ermöglichen und Anmerkungen zur lösung eventuell auftauchender
    Fehler zu veröffentlichen ist ein Benutzerhandbuch eine nützliche Ergänzung
    zur Anwendung. Obwohl die graphische Nutzeroberfläche (GUI) eine hohe
    Usability aufweisen soll und intuitiv zu bedienen sein muss, kann eine
    Anleitung in einigen Fällen eine extra Einweisung obsolet machen und Zeit
    sparen. Das Benutzerhandbuch soll in deutscher Sprache verfasst sein, den
    Aufbau der jeweiligen Ansichten der GUI erläutern und die funktionalität
    aller Bedienelemente erklären. Wenn nötig, soll es um eine Fehlerbehebungs-
    Sektion erweitert werden.\\
    \\
    \textbf{Anteil am Projektaufwandsvolumen: 3\%}


    \subsection{Rollentypen definitions Tool}

    Es soll eine Konfigurationsansicht erstellt werden mit der ein Nutzer neue 
    Rollentypen erstellen kann. Die Konfiguration des Rollentypus besteht aus
    der Zuweisung von Schreib-Privilegien, bestimmung der Ansichten in der
    Übersicht sowie der Namensgebung. Erstellte Rollentypen sollen dauerhaft in
    einer Konfigurationsdatei gespeichert werden. Erstellte Rollentypen sollen 
    wieder entfernt werden können.\\
    \\
    \textbf{Anteil am Projektaufwandsvolumen: 5\%}