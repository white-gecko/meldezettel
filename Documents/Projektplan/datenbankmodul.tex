\section{Datenbankmodul}
\textbf{Anteil am Projektaufwandsvolumen: 20\%}
\\ \\
Es wird ein Script geschrieben, das die Software `QuitStore' in einem
dedizierten Verzeichnis auf dem Zielcomputer installiert und konfiguriert.
Außerdem muss eine Bibliothek gefunden werden, die SPARQL Anfragen an die
Datenbank abstrahieren und formulieren kann.
Die eingegebenen Daten jedes Dokuments müssen in der Datenbank abgelegt werden.
Dafür ist es wichtig, jedem Vierfachvordruck eine eigene ID zuzuweisen, an welche detailierte Informationen (z.B Absender, Datum) im RDF-Format gekoppelt werden.
Diese Daten können über den im Quitstore enthaltenen SPARQL-Endpoint ausgelesen werden, um sie für weitere Auswertungen zu verwenden.
Mithilfe der Hash Funktionalität, die Git mit sich bringt und von QuitStore
genutzt wird um gespeicherte Einheiten kryptographisch zu markieren, kann die
Eindeutigkeit und die Unveränderbarkeit der Dokumente garantiert werden. 

