\section{Optimierung}
Sowohl die Effizienz als auch die Usability sind immens wichtige Kriterien für die fertige Anwendung. Um diese zu gewährleisten, müssen so viele Bugs wie möglich zum Ende der Entwicklungsphase eliminiert werden. Der Ermittlung dieser Bugs soll dabei zum Großteil durch die automatisierten Tests erfolgen um u.a. möglichst viel Zeit zu sparen und um deutlich mehr Fehler abzufangen. Die gefundenen Bugs werden danach im Quelltext gesucht und berichtigt. \\
\textbf{Anteil am Projektaufwandsvolumen: 5\%}