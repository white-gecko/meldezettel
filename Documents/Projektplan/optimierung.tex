\section{Optimierung}
\textbf{Anteil am Projektaufwandsvolumen: 10\%}\\ \\
Sowohl die Effizienz als auch die Usability sind sehr wichtige Kriterien für die Anwendung. Um diese zu gewährleisten, müssen so viele Bugs wie möglich zum Ende der Entwicklungsphase eliminiert werden. Die Ermittlung dieser Bugs soll dabei zum Großteil durch die automatisierten Tests erfolgen um u.a. Zeit zu sparen und um deutlich mehr Fehler abzufangen. Die gefundenen Bugs werden danach im Quelltext behoben. 
