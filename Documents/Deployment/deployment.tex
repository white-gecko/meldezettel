% !TeX spellcheck = de_DE
\documentclass[a4paper,11pt,oneside, titlepage]{article}
\usepackage[a4paper]{geometry} 
\geometry{a4paper,left=20mm, right=25mm, top=20mm, bottom=30mm} 
\usepackage[ngerman]{babel}
\usepackage{hyperref}
\usepackage[toc]{glossaries}
\usepackage[utf8x]{inputenc}
\usepackage[T1]{fontenc}
\usepackage{fancyhdr}
\usepackage{color}
\usepackage{dirtree}
\usepackage[]{listings} 

\pagestyle{fancy}

\lhead{\today}
\rhead{Verantwortlicher: Anja Sieke }
\chead{Gruppe: na17b}
\title{Installationsanleitung für Server}
\author{na17b}
\date{}

\definecolor{mygreen}{rgb}{0,0.6,0}
\definecolor{mygray}{rgb}{0.5,0.5,0.5}
\definecolor{mymauve}{rgb}{0.58,0,0.82}

\lstset{ 
  backgroundcolor=\color{white},   % choose the background color; you must add \usepackage{color} or \usepackage{xcolor}; should come as last argument
  basicstyle=\footnotesize,        % the size of the fonts that are used for the code
  breakatwhitespace=false,         % sets if automatic breaks should only happen at whitespace
  breaklines=true,                 % sets automatic line breaking
  commentstyle=\color{mygreen},    % comment style
  deletekeywords={...},            % if you want to delete keywords from the given language
  escapeinside={\%*}{*)},          % if you want to add LaTeX within your code
  extendedchars=true,              % lets you use non-ASCII characters; for 8-bits encodings only, does not work with UTF-8
  frame=single,	                   % adds a frame around the code
  keepspaces=true,                 % keeps spaces in text, useful for keeping indentation of code (possibly needs columns=flexible)
  keywordstyle=\color{blue},       % keyword style
  language=bash,                 % the language of the code
  morekeywords={*,...,sudo,node,npm,apt},            % if you want to add more keywords to the set
  numbers=none,                    % where to put the line-numbers; possible values are (none, left, right)
  numbersep=5pt,                   % how far the line-numbers are from the code
  numberstyle=\tiny\color{mygray}, % the style that is used for the line-numbers
  rulecolor=\color{black},         % if not set, the frame-color may be changed on line-breaks within not-black text (e.g. comments (green here))
  showspaces=false,                % show spaces everywhere adding particular underscores; it overrides 'showstringspaces'
  showstringspaces=false,          % underline spaces within strings only
  showtabs=false,                  % show tabs within strings adding particular underscores
  stepnumber=2,                    % the step between two line-numbers. If it's 1, each line will be numbered
  stringstyle=\color{mymauve},     % string literal style
  tabsize=2,	                   % sets default tabsize to 2 spaces
  title=\lstname                   % show the filename of files included with \lstinputlisting; also try caption instead of title
}

\begin{document}

  \maketitle

  \tableofcontents

  \section{Hinweise}

    Das Repository sollte direkt vom Server aus geklont werden, auf dem später
    die Software laufen soll. Auch wenn der Quelltext für die Webseite prinzipiell
    überall kompiliert und übertragen werden kann, so befinden sich im Repository jedoch
    einige Skripte, welche direkt auf dem Webserver ausgeführt werden müssen und ansonsten
    also ohnehin kopiert werden müssten. Dazu muss jedoch \verb+nodejs+ installiert sein.

    \begin{lstlisting}
      sudo apt install nodejs \end{lstlisting}

  \section{Vorbereitung des Servers}

    Die Software besteht serverseitig aus mehreren Komponenten. Dies ist zum
    einen der Webserver, welcher den Quelltext der Webanwendung zur Verfügung stellt,
    sowie ein Printservice, welcher übermittelte Daten in eine PDF-Datei umwandelt
    und zum Download anbietet, und zu guter Letzt der QuitStore, welcher die von den
    Anwendern eingegebenen Daten abspeichert.

    \subsection{Webserver}

      Damit Clients auf die Anwendung zugreifen können, muss diese von einem Webserver
      angeboten werden. Hierzu gibt es mehrere Möglichkeiten.

        \subsubsection{Apache}

          Dieser kann in allen gängigen Linux-Distributionen aus den offiziellen Paketquellen
          installiert werden. Unter Ubuntu/Debian/Raspbian geschieht dies zum Beispiel
          folgendermaßen.

          \begin{lstlisting}
            sudo apt install apache2 \end{lstlisting}

				\subsubsection{Via Node-Skript}

          Sollte obiges keine Option sein, befindet sich im Ordner \verb+frontend/src+ ein Skript
          namens \verb+server.js+.

          Dieses bindet im Normalfall auf Port 8080 und wird via \verb+nodejs+ ausgeführt. Dies geschieht
          jedoch erst, nachdem der Quelltext kompiliert wurde.

          \begin{lstlisting}
            node server.js \end{lstlisting}

      \subsection{Printservice}

        Der Printservice verwendet Python3, um ein tex-Template auszufüllen und dieses anschließend
        zu kompilieren. Die dafür notwendigen Pakete lassen sich wie folgt installieren.

        \begin{lstlisting}
          sudo apt install texlive texlive-lang-german texlive-doc-de \
            texlive-latex-extra python3 \end{lstlisting}

        Um diesen zu starten, genügen folgende Befehle von der Wurzel des Repositories aus.

        \begin{lstlisting}
          cd Projekt/src/renderPDF
          python3 server.py \end{lstlisting}

      \subsection{QuitStore}

        Für die Installation und das Starten des QuitStores via Docker liegt ein Skript bereit,
        dieses muss lediglich ausgeführt werden.

        \begin{lstlisting}
          cd Projekt
          chmod +x quitstore.sh
          ./quitstore.sh \end{lstlisting}

        Alternativ dazu wurde auch ein npm-Skript angelgt, welches den QuitStore bei Bedarf installiert
        und startet.

        \begin{lstlisting}
          cd Projekt/src/frontend
          npm run quit \end{lstlisting}

  \section{Erstellen des Quelltextes}

    Nachdem nun alle Dienste laufen, fehlt noch der eigentliche Quelltext der Webapp. Zur Steigerung
    der Performance muss dieser nach der Konfiguration kompiliert werden und durchläuft dabei mehrere Optimierungsphasen.

    \subsection{Konfiguration}

      Die folgenden Schritte beziehen sich auf \verb+Projekt/src/frontend+ als Wurzelverzeichnis.

      Zunächst muss der Pfad angepasst werden, unter dem die Anwendung später zu erreichen sein wird.
      Soll sie zum Beispiel unter der URL \verb+my.domain.de/thw+ erreichbar sein, so lautet der Pfad
      \verb+/thw+. In den meisten Fällen wird dies jedoch schlicht \verb+/+ sein. Die Option hierfür befindet
      sich in \verb+config/index.js+ und lautet \verb+assetsPublicPath+ im Unterpunkt \verb+build+ und sollte
      sich im Bereich um Zeile 53 befinden.

      Als zweites muss der Anwendung der Name des Servers mitgeteilt werden, auf dem Printservice und QuitStore laufen.
      Die Option hierfür befindet sich in den Dateien \verb+PrintServiceAdapter.js+ sowie \verb+QuitStoreAdapter.js+ im
      Unterordner \verb+src/api+ und heißt in beiden Fällen \verb+url+.

      Zum Schluss befinden sich im Ordner \verb+src/config+ noch einige Konfigurationsdateien, welche das Verhalten
      der Anwendung und damit das Nutzererlebnis beeinflussen. Diese können nach Belieben angepasst werden, sind aber
      bereits mit sinnvollen Standardwerten belegt.

    \subsection{Kompilieren}

      Nachdem die Konfiguration abgeschlossen ist, müssen noch die Abhängigkeiten des Quelltextes heruntergeladen werden.
      Dies vermeidet lästige Downloads von externen Webseiten beim Aufruf der Anwendung, welche das Laden stark verzögern
      oder sogar unmöglich machen können.

      \begin{lstlisting}
        cd Projekt/src/frontend
        npm install \end{lstlisting}

      Dieser Vorgang dauert je nach Internetverbindung mehrere Minuten, da etwa 300MB an Paketen heruntergeladen werden müssen.
      Anschließend kann der Quelltext kompiliert werden.

      \begin{lstlisting}
        npm run build \end{lstlisting}

      Die fertigen Dateien befinden sich nun im Ordner \verb+Projekt/src/frontend/dist+.

    \subsection{Bereitstellen des Quelltexts}

      Wurde als Webserver Apache gewählt, so genügt es, den \textbf{Inhalt} des Ordners \verb+dist+, also
      die Datei \verb+index.html+ sowie den Ordner \verb+static+ zu kopieren. Diese landen nun
      beispielsweise direkt im Wurzelverzeichnis von Apache, unter Debian und Derivaten also in \verb+/var/www/html+.

      Wurde kein Webserver installiert, so muss die Datei \verb+server.js+ auf den Server ausgeführt werden. Zu beachten ist 
      dann aber, dass die Anwendung nicht unter Port 80 erreicht werden kann, sondern stets Port 8080 in die 
      Adressleiste des Browsers mit eingegeben werden muss.

\end{document}
