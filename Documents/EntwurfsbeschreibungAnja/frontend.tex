\section{Struktur einzelner Pakete}
\label{sec:struktur_einzelner_pakete}

  \subsection*{Frontend}
  \label{sub:frontend}

    Das Frontend ist in \gls{Javascript} geschrieben und verwendet das \gls{Vue.js}-Framework. Als
    Erweiterungen kommen \gls{Vuex}, sowie \gls{vue-router} zum Einsatz. Die Verwaltung des Quelltextes
    sowie das kompilieren geschieht via \gls{webpack}. Zur Vereinfachung wird auf die
    Bibliothek von \gls{Element-UI} zurückgegriffen.
  
    Der Quelltext des Frontends befindet sich im Ordner \verb+src/frontend+. In diesem Ordner
    befinden sich neben einem weiteren Ordner \verb+src+ die von \gls{webpack} benötigten Dateien
    und Ordner; diese werden im Folgenden nicht weiter beschrieben und sind weitgehend Standard.

    Der Ordner \verb+src+ unterteilt sich weiter in folgende Struktur:

    \dirtree{%
      .1 ./src/frontend/src/.
        .2 assets/.
        .2 components/.
        .2 router/.
        .2 store/.
          .3 state.js.
          .3 getters.js.
          .3 mutations.js.
          .3 index.js.
        .2 App.vue.
        .2 main.js.
    }

    \begin{itemize}
      \item \verb+assets+ Mediendateien (aktuell nur das THW-Logo)
      \item \verb+components+ Vue-Komponenten
      \item \verb+router+ Konfiguration von vue-router
      \item \verb+store+ Konfiguration von Vuex
        \begin{itemize}
          \item \verb+state.js+ deklariert globale Zustandsvariablen
          \item \verb+getters+ stellt Funktionen zum Abrufen der Variablen bereit
          \item \verb+mutations+ enthält Funktionen zum Bearbeiten des Zustandes
          \item \verb+index.js+ fügt die obigen Dateien zusammen zu einem globalen Store
        \end{itemize}
      \item \verb+App.vue+ Wrapper-Komponente für das Frontend
      \item \verb+main.js+ Einstiegspunkt für das Programm, enthält alle Importe
    \end{itemize}
    
