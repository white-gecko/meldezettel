\newglossaryentry{Element-UI} {
    name = Element-UI,
    description = {
        Bibliothek vorgefertigter Komponenten für Vue.js
    }
}

\newglossaryentry{Javascript} {
    name = JavaScript,
    description = {
        Skriptsprache, die hauptsächlich auf Webseiten Anwendung findet und zumeist clientseitig ausgeführt wird, um den Server zu entlasten.
    }
}

\newglossaryentry{JSON} {
    name = JSON,
    description = {
        JSON steht für Javascript Object Notation und ist ein Standard, der Datenspeicherung für einfachen Datenaustausch ermöglichen soll.
    }
}

\newglossaryentry{Quitstore} {
    name = Quitstore,
    description = {
        Eine Software, die Git-Versionierung für sogenannte `Named Graphs' (ein Schlüsselkonzept des Semantic Web das bewirkt, dass eine Menge an RDF Ausdrücken über eine URI identifiziert werden) ermöglicht.
    }
}

\newglossaryentry{RDF} {
    name = RDF,
    description = {
        RDF steht für Resource Description Framework. RDF ist ein Datenmodell, welches alle Daten in Triples der Form Subjekt,Prädikat, Objekt dargestellt.
    }
}

\newglossaryentry{SPARQL} {
    name = SPARQL,
    description = {
        SPARQL steht für Semantic Protocol and RDF Query Language. SPARQL ist eine Anfragesprache, mit der man Informationen aus einem RDF Datenmodell extrahieren und modifizieren kann.
    }
}

\newglossaryentry{vue-router} {
    name = Vue-Router,
    description = {
        Ermöglicht die Verwendung der Browser-Historie, sowie das gezielte Anzeigen von Komponenten abhängig vom Pfad.
    }
}

\newglossaryentry{Vue.js} {
    name = Vue.js,
    description = {
        Reaktives JavaScript-Framework zur einfachen Erstellung interaktiver Frontends.
    }
}

\newglossaryentry{Vuex} {
    name = Vuex,
    description = {
        Erweitert Vue.js um einen globalen Speicher für Zustandsvariablen, sowie Funktionen zum einfacheren Bearbeiten und Methoden zum Debuggen.
    }
}

\newglossaryentry{webpack} {
    name = webpack,
    description = {
        Build-Tool für Webseiten, hauptsächlich für JavaScript. Erlaubt das Verwenden nahezu beliebiger Dateiformate und Sprachstandards für den Entwickler und erzeugt eine mit allen gängigen Browsern kompatible und minimierte Webseite.
    }
}
