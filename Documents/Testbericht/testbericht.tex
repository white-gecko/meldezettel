\documentclass[a4paper,11pt,oneside, titlepage]{article}
\usepackage[a4paper]{geometry} 
\geometry{a4paper,left=25mm, right=25mm, top=20mm, bottom=30mm} 
\usepackage[ngerman]{babel}
\usepackage[utf8x]{inputenc}
\usepackage[T1]{fontenc}
\usepackage{fancyhdr}
\usepackage{hyperref}
\usepackage{graphicx}
\usepackage[toc]{glossaries} 

\renewcommand{\arraystretch}{2}
\renewcommand\thesubsection{}

\pagestyle{fancy}

\lhead{\today}
\rhead{Verantwortliche: Anja Sieke}
\chead{Gruppe: na17b}
\title{Testbericht\\Nachrichtenkommunikation für das THW}
\author{na17b}
\date{}

\makeglossaries

\begin{document}
  
  \maketitle

  \tableofcontents

  \newpage

  \section{Allgemeines}
  \label{sec:allgemeines}

    Das Vorprojekt führt ausschließlich Komponententests durch, da die Kommunikation
    zwischen Komponenten nur indirekt über den Store erfolgt. Der Zugriff auf den Store wurde
    gemocked.

    Als Testframework wird \gls{Jest} zusammen mit \gls{vue-test-utils} verwendet.
  
  \section{Tests}
  \label{sec:tests}

    \subsection{Komponententests}
    \label{sub:komponententests}

      Die Testspezifikationen befinden sich vom frontend-Verzeichnis aus gesehen in \verb+test/unit/specs+.
      Dort liegt für jede Komponente eine eigene Datei, welche die geforderten Eigenschaften und Funktionen
      einer Komponente beschreibt.

      Es genügt, im Ordner frontend den Befehl \verb+npm run unit+ auszuführen; daraufhin werden alle
      Test-Suites automatisch abgearbeitet. Eine beispielhafte Ausgabe ist in folgender Abbildung zu sehen.

      \begin{figure}[htpb]
        \centering
        \includegraphics[width=0.8\linewidth]{testsscreenshot}
        \caption{Ausgabe des Befehls npm run unit}
        \label{fig:npmtest}
      \end{figure}

      Die Tests beschränken sich hier zunächst auf das Prüfen von Anwesenheit bestimmter Variablen und html-Elementen.
      Die Ergebnisse sind in folgender Tabelle aufgelistet.

      \begin{table}[htpb]
        \centering
        \label{tab:test}
        \begin{tabular}{c | c | c}
          Komponente & Anzahl Tests & Bestanden \\
          \hline
          THWForm & 5 & ja \\
          THWDashboard & 4 & ja \\
          THWMenu & 3 & ja 
        \end{tabular}
        \caption{Testergebnisse der momentanen Frontend-Komponenten}
      \end{table}

      \newpage

      \newglossaryentry{elementui} {
    name = Element-UI,
    description = {
        Bibliothek vorgefertigter Komponenten für Vue.js
    }
}

\newglossaryentry{javascript} {
    name = JavaScript,
    description = {
        Skriptsprache, die hauptsächlich auf Webseiten Anwendung findet und zumeist clientseitig ausgeführt wird, um den Server zu entlasten.
    }
}

\newglossaryentry{vuerouter} {
    name = Vue-Router,
    description = {
        Ermöglicht die Verwendung der Browser-Historie, sowie das gezielte Anzeigen von Komponenten abhängig vom Pfad.
    }
}

\newglossaryentry{vuejs} {
    name = Vue.js,
    description = {
        Reaktives JavaScript-Framework zur einfachen Erstellung interaktiver Frontends.
    }
}

\newglossaryentry{vuex} {
    name = Vuex,
    description = {
        Erweitert Vue.js um einen globalen Speicher für Zustandsvariablen, sowie Funktionen zum einfacheren Bearbeiten und Methoden zum Debuggen.
    }
}

\newglossaryentry{webpack} {
    name = webpack,
    description = {
        Build-Tool für Webseiten, hauptsächlich für JavaScript. Erlaubt das Verwenden nahezu beliebiger Dateiformate und Sprachstandards für den Entwickler und erzeugt eine mit allen gängigen Browsern kompatible und minimierte Webseite.
    }
}

      \printglossaries
  
\end{document}
