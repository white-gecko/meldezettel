\documentclass[a4paper,11pt,oneside, titlepage]{article}
\usepackage[a4paper]{geometry} 
\geometry{a4paper,left=25mm, right=25mm, top=20mm, bottom=30mm} 
\usepackage[ngerman]{babel}
\usepackage[utf8x]{inputenc}
\usepackage[T1]{fontenc}
\usepackage{fancyhdr}
\usepackage{hyperref}
\usepackage{graphicx}
\usepackage{verbatim}
\usepackage[toc]{glossaries} 

\renewcommand{\arraystretch}{2}
\renewcommand\thesubsection{}

\pagestyle{fancy}

\lhead{\today}
\chead{Gruppe: na17b}
\rhead{Verantwortlicher: Marc Wahsner}
\title{Testbericht\\Nachrichtenkommunikation für das THW}
\author{na17b}
\date{}

\makeglossaries
\newglossaryentry{elementui} {
    name = Element-UI,
    description = {
        Bibliothek vorgefertigter Komponenten für Vue.js
    }
}

\newglossaryentry{javascript} {
    name = JavaScript,
    description = {
        Skriptsprache, die hauptsächlich auf Webseiten Anwendung findet und zumeist clientseitig ausgeführt wird, um den Server zu entlasten.
    }
}

\newglossaryentry{vuerouter} {
    name = Vue-Router,
    description = {
        Ermöglicht die Verwendung der Browser-Historie, sowie das gezielte Anzeigen von Komponenten abhängig vom Pfad.
    }
}

\newglossaryentry{vuejs} {
    name = Vue.js,
    description = {
        Reaktives JavaScript-Framework zur einfachen Erstellung interaktiver Frontends.
    }
}

\newglossaryentry{vuex} {
    name = Vuex,
    description = {
        Erweitert Vue.js um einen globalen Speicher für Zustandsvariablen, sowie Funktionen zum einfacheren Bearbeiten und Methoden zum Debuggen.
    }
}

\newglossaryentry{webpack} {
    name = webpack,
    description = {
        Build-Tool für Webseiten, hauptsächlich für JavaScript. Erlaubt das Verwenden nahezu beliebiger Dateiformate und Sprachstandards für den Entwickler und erzeugt eine mit allen gängigen Browsern kompatible und minimierte Webseite.
    }
}


\begin{document}

\maketitle

\tableofcontents

\newpage

\section{Allgemeines}
Die Tests für dieses Softwareprojekt sind eng an das Qualitätssicherungskonzept geknüpft und sollen das Funktionieren des Codes gewährleisten. Das Vorprojekt führt ausschließlich Komponententests durch, da die Kommunikation zwischen Komponenten nur indirekt über den Store erfolgt. Der Zugriff auf den Store wurde gemocked. Als Testframework wird \gls{Jest} zusammen mit \gls{vue-test-utils} verwendet.
\section{Tests}
\subsection{Komponententests}
Komponententests befinden sich jeweils im gleichen Ordner wie die zu testende Komponente.
\subsection{GUI-Tests}
Die Testspezifikationen befinden sich vom frontend-Verzeichnis aus gesehen in \verb+test/unit/specs+.
Dort liegt für jede Komponente eine eigene Datei, welche die geforderten Eigenschaften und Funktionen
einer Komponente beschreibt. Es genügt, im Ordner frontend den Befehl \verb+npm run unit+ auszuführen; daraufhin werden alle Test-Suites automatisch abgearbeitet. Eine beispielhafte Ausgabe ist in folgender Abbildung zu sehen.
\begin{figure}[htpb]
\centering
\includegraphics[width=0.8\linewidth]{test}
\caption{Ausgabe des Befehls npm run unit}
\label{fig:npmtest}
\end{figure}
Die Tests umfassen das Prüfen von Anwesenheit bestimmter Variablen und html-Elemente, sowie die Rückgabewerte einiger Funktionen und das Parsen von sparql-Rückgaben.
Die Ergebnisse sind in folgender Tabelle aufgelistet.
\begin{table}[htpb]
\centering
\label{tab:test}
\begin{tabular}{c | c | c}
Komponente & Anzahl Tests & Bestanden \\
\hline
THWDashboard & 4 & ja \\
THWMenu & 3 & ja \\
THWLandingPage & 2 & ja\\
sparql\_response & 1 & ja\\
sparql\_queries & 3 & ja\\
\end{tabular}
\caption{Testergebnisse der momentanen Frontend-Komponenten}
\end{table}
Die Tests für THWForm werden derzeit nicht ausgeführt, weil es zu Komplikationen mit Jest kam, welche noch nicht behoben werden konnten.
\section{Continous Integration}
\label{sub:continous integration}
Zum zweiten Release wurde Continous Integration eingeführt, um das Einhalten der Vorgaben aus dem Dokumentationskonzept und Coding Standards automatisiert zu testen. Dazu wird GitLab CI \gls{GitLab-CI} verwendet. Die \verb+gitlab-ci.yml+ beinhaltet XML-Linting sowie JavaScript-Linting und -Testing. Um den Prozess zu beschleunigen wird auf \verb|npm install| verzichtet, stattdessen werden nur für die Tests benötigte Pakete und deren Abhängigkeiten installiert. Anschließend erfolgen linting (\verb|npm run lint|) und testing (\verb|npm run unit|).

\newpage
\printglossaries
\end{document}
