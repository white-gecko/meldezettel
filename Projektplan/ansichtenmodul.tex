\section{Ansichtenmodul} 
Das Arbeitspaket 'Ansichtenmodul' beinhaltet alle Frontend-Funktionalitäten. 
Dabei wird zwischen den drei Bereichen statische Sichten, dynamische Sichten 
und Prozessschritt unterschieden. 
\subsection{statische Sichten}
Der Bereich statische Sichten beinhaltet mit der Landing Page inklusive 
Kürzeleingabe und Rollenauswahl die Ansicht, die rollenunabhängig für alle User sichtbar ist.
\subsection{dynamische Sichten}
Im Bereich dynamische Sichten werden alle rollenspezifischen Sichten 
behandelt: Dabei wird anhand der Rollen Funker, Sichter, Leiter und 
Sachbearbeiter unterschieden, welche Funktionalitäten das Dashboard jeweils 
aufweisen muss. Funker müssen Nachrichten erstellen und empfangen können.
Sichter müssen Nachrichten empfangen, weiterleiten, zur Überarbeitung 
zurückweisen und mit Notizen versehen können. Die Sachbearbeiter müssen 
Nachrichten erstellen, empfangen und drucken können. 
Allgemein muss für alle Rollen gewährleistet sein, dass sie die 
Nachrichtenhistorie einsehen können.
\subsection{Prozessschritt} 
Der Bereich Prozessschritt beinhaltet den Satus einer Nachricht 
auf einem vorliegenden Vierfachvordruck. Die Möglichkeit zur Statustabfrage 
soll durch Dreierlei ermöglicht sein: Die Farbe des vorliegenden Dokuments, 
in Orientierung am originalen Vierfachvordruck, Einblendung eines 
Textes zum Status der Nachricht sowie die o.g. Möglichkeit zur Einsicht der 
Historie der Nachricht.
\textbf{Anteil am Projektaufwandsvolumen: 5\%}