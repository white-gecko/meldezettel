\subsection{gitlab}
GitLab ist eine Webanwendng zur Versionsverwaltung für Softwareprojekte auf Git-Basis und bietet sowol diverse Management und Bug-Tracking-Features, als auch ein System für kontinuierliche Integration. GitLab ist dabei offen für viele verschiedene Arten der Softwareentwicklung (Waterfall, Agile Development,...), da die Management-tools wie z.B. das Issue-Board sehr frei einsetzbar sind. Weitere wichtige Planungs/Management-Tools sind z.B. Time-Tracking, Milestones, Issues und die Zuweisung zu spezifischen Personen. Auch die Entwicklung an sich Gestaltet sich sehr komfortabel, vor allem durch die visuelle Representation von Branches, Merge Requests uvm. Des Weiteren lässt sich sehr klar definieren, wie der Arbeitsablauf aussieht, z.B. Sperrung der Push-Funktion auf den Master-Branch. Somit kann sicher gestellt werden, das keine Fehler entstehen und sich jeder auf die eigentliche Aufgabe konzentrieren kann, anstatt sich mit dem Managment des Projektes zu beschäfftigen.
\cite{gitlabwiki}	
