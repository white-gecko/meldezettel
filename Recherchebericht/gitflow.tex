\subsection{gitflow}
Gitflow ist ein Konzept für einen Arbeitsablauf mit Git. Dabei wird vorallem die Geschichte sehr gut strukturiert und bleib Übersichtlich auch für Aussenstehende.

\noindent Das Konzept basiert auf zwei Hauptverzweigungen, zum einem dem Master-Branch zum anderen dem Develop-Branch, und kleinen Feature- und Release-Branches. Auf den Master-Branch werden dabei nur fertige Releases aus den Release-Branches commited. Der Develop-Branch ist der Zweig für die Entwicklung der einzelnen Releases. In ihn werden die einzelnen Feature-Branches gemerged. Feature Branches sind dabei kleine Zweige, welche für jedes zu implementierende Feature erstellt werden. Sie werden in den Develop-Branch gemerged, wenn das Feature fertig implementiert ist. Ein Release-branch wird immer dann aus Develop eröffnet, wenn ein Release ansteht und genug Features im Develop-branch implementiert sind. In ihm werden keine neuen Features mehr hinzugefügt, sondern nur noch Bug-Fixes oder Dokumentation.

\noindent Dieses Vorgehen bietet vorallem viel Sicherheit und Struktur.
\cite{gitflow}	
