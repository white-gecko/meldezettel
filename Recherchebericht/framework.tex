\subsection{Framework}
Ein Framework stellt dem Softwareentwickler einen Rahmen mit festen Regeln und wiederverwendbare Strukturen zur Verfügung. Insbesondere definiert es den Kontrollfluss der Anwendung und die Schnittstellen der Klassen. Im Gegensatz zur Arbeit mit Klassenbibliotheken werden nicht bereits vorhandene Klassen und Funktionen verknüpft und verwendet um den Ansprüchen des Projekts zu genügen, stattdessen werden selbständig Funktionen und Klassen in einem vorgegebenen Rahmen implementiert, welche dann vom Framework genutzt werden um vordefinierte Aufgaben zu erledigen. Woraus sich auch ergibt das Frameworks stark Anwendungsspezifisch sind. \cite{frameworkwiki}