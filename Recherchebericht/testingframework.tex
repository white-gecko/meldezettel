\subsection{Testing Framework}
Ein Testing Framework bezeichnet ein Framework zum automatisierten Softwaretesting. Insbesondere in der agilen Softwareentwicklung mit häufigen Anspruchsänderungen und den einhergehenden Anpassungen einzelner Module, ist eine schnelle, fehlerfreie und automatisierte Funktionalitätsprüfung geeignet. Aber auch um zu überprüfen ob eine Anpassung nötig ist oder das Modul auch den geänderten Ansprüchen standhält kann Softwaretesting verwendet werden. Desweiteren haben somit auch mit dem jeweiligen Modul weniger vertraute Teammitglieder die Möglichkeit schnell verlässliche Antworten auf zahlreiche Fragestellungen zu erhalten. Beispielsweise ist es möglich zu überprüfen ob eine nachträgliche Detailänderung die Funktionalität eines Moduls beeinträchtigt. Aber auch während der Entwicklung erlaubt das Software Testing kontinuierliche Funktionalitätstests und damit eine frühzeitige, unkomplizierte und effiziente Fehlererkennung und Minimierung. Frameworks sind strukturell bestens für die Vereinfachung des Softwaretesting geeignet, weil sie das Definieren der Ein- und Ausgabeparameter in einer bestehenden und durchdachten Struktur erlauben und das eigentliche Testen der Software übernehmen können. Wichtige Parameter für ein solches Testing Framework sind gute Dokumentation, gute Bedienbarkeit insbesonder in Bezug auf die Nutzung mehrerer Parametersätze und präzise Fehlerauswertung. \cite{modultestwiki} \cite{frameworkwiki}