\subsection{Digitalisierung von Arbeitsabl�ufen}
In einem sich rapide weiterentwickelnden Informations- und Digitalisierungszeitalter m�chte jeder innerhalb von Sekundenbruchteilen auf Nachrichten und Daten von �berall aus zugreifen k�nnen. Der Umstieg von Papier zu Kupferdraht und Glasfaser erm�glicht genau das. Dies ist auch von hoher Relevanz f�r das technische Hilfswerk, da Informationen wie der Vierfachvordruck nicht manuell von einer Person zur anderen transportiert werden m�ssen, sondern digital ein Formular ausgef�llt wird, welches dann �ber das Internet verschickt wird und eine Sekunde sp�ter beim Empf�nger ausgedruckt werden kann. Dadurch wird nicht nur Schnelligkeit, sondern auch eine gewisse Sicherheit gew�hrleistet. Papier ist leichter zu verlieren bzw. zu zerst�ren als digitale Daten. Au�erdem sind Entfernungen absolut kein Problem beim digitalen Versenden, der Zeitunterschied von 20km zu 200 km betr�gt ca. 1 Sekunde, w�hrend beim manuellen Verschicken zwischen 30 Minuten und 3 Stunden unterschieden werden muss.