\subsection{Digitalisierung von Arbeitsabläufen}
In einem sich rapide weiterentwickelnden Informations- und Digitalisierungszeitalter möchte jeder innerhalb von Sekundenbruchteilen auf Nachrichten und Daten von überall aus zugreifen können. Der Umstieg von Papier zu Kupferdraht und Glasfaser ermöglicht genau das. Dies ist auch von hoher Relevanz für das technische Hilfswerk, da Informationen wie der Vierfachvordruck nicht manuell von einer Person zur anderen transportiert werden müssen, sondern digital ein Formular ausgefüllt wird, welches dann über das Internet verschickt wird und eine Sekunde später beim Empfänger ausgedruckt werden kann. Dadurch wird nicht nur Schnelligkeit, sondern auch eine gewisse Sicherheit gewährleistet. Papier ist leichter zu verlieren bzw. zu zerstören als digitale Daten. Außerdem sind Entfernungen absolut kein Problem beim digitalen Versenden, der Zeitunterschied von 20km zu 200 km beträgt ca. 1 Sekunde, während beim manuellen Verschicken zwischen 30 Minuten und 3 Stunden unterschieden werden muss.